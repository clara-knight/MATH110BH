\documentclass{notes}
\usepackage{myalgebra}

\begin{document}
\justifying 
\maketitle

%\tableofcontents
\section{Week 1}
\subsection{Rings and examples of rings}
Consider a field $F$ and a vector space $V$ over $F$.
Then $\mathcal{L}(V)$ itself is not a vector space:
If $S$, $T \in \mathcal{L}(V)$ then it is not always the case that 
$T \circ S = S \circ T$, and it is not the case that for every 
$T \in \mathcal{L}(V)$ there is a $T^{-1}$ such that
$T \circ T^{-1} = I$.
Then $\mathcal{L}(V)$ is a ring:

\begin{defn}[Ring] 
	A ring $(R, +, \cdot)$ is a set $R$ with two operations,
	denoted $+$ and $\cdot$ such that
	for all $a$, $b$, $c$ in $R$:
	\begin{enumerate}[(i)]
		\item $(R, +)$ is an abelian group.
		\item $(ab)c = a(bc)$.
		\item $a(b+c) = ab+ac$ and $(b+c)a = ba+ca$.
	\end{enumerate}
	If $ab = ba$ for all $a$, $b \in R$ then $R$ is a commutative ring. 
	If $1_{R}a = a = a 1_{R}$ for all $a \in R$ then $R$
	is a ring with identity.
\end{defn}

\begin{exmp}
	$(\Z, +, \cdot)$ is a commutative ring with identity.
\end{exmp}
\begin{exmp}
	$(\mathcal{L}(V), +, \circ)$ is a ring with identity
\end{exmp}

A nonzero element $a \in R$ is a left (right) zero divisor if there is
a nonzero element $b \in R$ such that $ab = 0$ ($ba = 0$).
A zero divisor is both a left and right zero divisor.

An element $a$ in a ring $R$ with identity is said to be left (right)
invertible if there exists  a $c\in R$ such that $ca = 1_R$ ($ac=1_R$).
The element $c$ a left (right) inverse of $a$. An element $a \in R$ that
is both left and right invertible is invertible or a unit.

The set of units in a ring $R$ with identity forms a group under multiplication.

There are rings with identity and inverses that do not commute:
\begin{exmp}
	$\mathbb{H}$, the quaternions, is a non-commutative 
	ring with identity and inverses, also known as
	division ring or skew field.
\end{exmp}

Consider $\mathcal{F}(X, R)$, where $X \neq \varnothing$ and
$R$ is a ring. This is the set of functions from $X$ to $R$.
Then for $f, g$ we can define addition as 
$(f+g)(x)  = f(x)+g(x)$ and multiplication as 
$(f+g)(x)  = f(x)g(x)$. Then $\mathcal{F}(X, R)$ is a ring.

\begin{defn}[Integral domain]
	A commutative ring $R$ with identity $1_R \neq 0$ and no zero ddivisors
	is called an integral domain. A ring $D$ with identity  $1_D \neq 0$ in
	which every nonzero element is a unit is called a division ring.
	A field is a commutative division ring.
\end{defn}

\subsection{Ring homomorphisms}

Let $R$, $S$ be rings and $\phi: R \rightarrow S$ a function such that
for all $a$, $b \in R$,
$\phi(a+b) = \phi(a) + \phi(b)$ and $\phi(ab)=\phi(a)\phi(b)$.

\end{document}
